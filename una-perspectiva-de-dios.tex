\documentclass[letterpaper,12pt]{article}

\usepackage[utf8]{inputenc}    % Codificación de caracteres
\usepackage[spanish]{babel}    % Idioma español
\usepackage{amsmath}           % Para matemáticas
\usepackage{amsfonts}          % Para fuentes matemáticas
\usepackage{amssymb}           % Para símbolos matemáticos

\title{UNA PERSPECTIVA DE DIOS}
\author{Rosendo Camal}
\date{28 de diciembre de 2024}

\begin{document}

\maketitle

Las matemáticas sirven para todo y al mismo tiempo, para nada. Se han mantenido estas dos respuestas y ambas son correctas, no es una ambigüedad ni una contradicción. La misma matemática da respuesta a esta cuestión, a través de la Teoría de Conjuntos. El conjunto vacío (\(\{\emptyset\}\)) es subconjunto de cualquier y de todos los conjuntos. Esto le podemos dar la equivalencia y a reducir que la NADA forma parte del TODO, siendo la nada única y exclusivamente la ausencia del todo. El vacío no es un elemento activo en sí, sino el surgimiento inevitable en la ausencia del todo. Por lo que, el conjunto vacío -la nada- se encuentra en todos los conjuntos -el todo-.

Al sexto día, Dios creó al ser humano a imagen y semejanza de Él. Es cierto, somos casi iguales a Él y también por ello, somos distintos de Dios porque ser casi iguales no es lo mismo que ser iguales o ser lo que es. A lo largo de la Biblia y de la historia, Dios se ha manifestado actuando semejante a nosotros o que su actuar sea semejante al nuestro. De todas esas acciones, hay una en particular. Dios se arrepintió de haber creado al ser humano. Los humanos también nos arrepentimos de nuestros actos pasados, sean correctos o incorrectos, buenos o malos. Dios se arrepintió de su creación, esto quiere decir mucho de Él. No es muy diferente a nosotros, solo se diferencia porque Él es infinito y divino, y nosotros lo finito y limitado. También corrigió su error, enviando el Diluvio Universal, así como nosotros tratamos o corregimos nuestros actos.

Dios es aproximadamente igual a nosotros y viceversa, pero no somos iguales. Ser igual no es lo mismo que ser aproximadamente igual o casi igual, por lo que lo igual y lo aproximadamente igual son distintos.

\[
(=) \neq (\approx)
\]

Daré una ilustración categórica de Dios y a nosotros los humanos, es una forma de presentar a Dios bajo mi perspectiva aunque considero que podría no ser la mejor y podría tener muchas imprecisiones y debilidades al momento de abordar estas cuestiones a pesar de la sencillez.

Primero, definamos a Dios y a los humanos.

\[
\text{Dios} := \pi
\]
\[
\text{Humanos} := 3.14 \quad \text{o} \quad \text{Humanos} := 3.1416
\]

Dios lo podemos representar como aquellos números irracionales que tanto nos asombran, tales como el ya mencionado \(\pi\), \(e\), \(\varphi\) o \(\lambda\). Estos pertenecen al conjunto de los números irracionales (\(I\)). ¿Qué tiene de especial \(\pi\)? Me sirve para representar lo infinito y lo divino, mientras que las aproximaciones como \(3.14\) o \(3.1416\) y todas las demás aproximaciones que hemos logrado obtener de \(\pi\) como sociedad son casi iguales, pero no son iguales, porque siguen siendo aproximaciones y son finitos y limitados.

Podemos comprender que tanto \(\pi \approx 3.14\) como \(\pi \approx 3.1416\) son solo aproximaciones, así es Dios es aproximadamente igual a los humanos y viceversa. Pero \(\pi \neq 3.14\) y \(\pi \neq 3.1416\) a pesar de ser casi iguales, también Dios es distinto de nosotros. Por lo que \(\pi\) y solo \(\pi\) puede ser igual a \(\pi\).

\[
\exists! x (x = \pi) \Rightarrow x = \pi \quad \text{y} \quad \pi = \pi \Rightarrow \pi \text{ y solo } \pi \text{ es igual a } \pi
\]

La expresión significa que existe una única cosa o existe exactamente una cosa que sea igual a \(\pi\) y solo \(\pi\) satisface esa igualdad. No queda más que decir de la siguiente afirmación: \(\exists! x (x = \pi)\). Por lo que \(3.14\), \(3.1416\) y las demás aproximaciones son números pertenecientes al conjunto de los números racionales (\(Q\)) y todas tienen estas aproximaciones son finitas y limitadas, no son iguales por más que se aproximen a ellas. Por tanto, Dios y solo Dios es igual a Dios, por mucho que nosotros seamos semejantes y a imagen de Él. Hay una característica o un conjunto de características que separa al Creador de su Creación. Esto es la esencia del infinito y lo divino, con lo finito y lo limitado. A Dios le corresponde lo primero, y a nosotros los mortales lo finito y limitado.

Con todo lo anterior, no estoy mencionando que \(\pi\) es igual o es Dios. Es una clasificación ilustrativa de utilidad comparativa y considero bastante imprecisa con limitaciones para abarcar esta cuestión.

De igual manera, Dios es el TODO, como el conjunto universal (\(U\)) y su creación es parte de Él, el conjunto (\(A\)). Siendo parte de Él la NADA, el conjunto vacío (\(\{\emptyset\}\)). Por lo que la maldad es parte de Dios, como es parte de los humanos. Esto no representa si Dios es malo, sino que el conjunto vacío (\(\{\emptyset\}\)) se puede interpretar como la maldad. Esto es inevitable, esto es así ante la ausencia de Dios. El conjunto vacío tiene como naturaleza ser el subconjunto de todos los conjuntos, el conjunto vacío la nada y todos los conjuntos como el conjunto universal (\(U\)) son el todo.

La maldad es la ausencia de Dios, como la percepción limitada de nuestros ojos ante la oscuridad descrita como ausencia de luz (siendo una limitante biológica de nuestros sentidos), es un desorden o desviación de lo divino, así la maldad y la nada no son un elemento activo o no son un elemento en esencia de lo que deberían llegar a ser, sino porque no llegan y no llegarán a ser.

El vacío al ser parte del todo y a la vez no ser nada, no implica que no haya existencia solo marca la inexistencia de la existencia. La maldad forma parte de Dios, no porque sea malo, carezca de bondad o sea incapaz de ser plenamente bondadoso sino porque al ser algo más allá de los límites de la libertad y el libertinaje, quizás el libre albedrío, su creación y Él pueda contener a la maldad. Su creación puede divergir de lo divino y de igual forma, converger en lo divino si así lo desea la propia creación. La maldad inevitable por su naturaleza ante la ausencia de Dios, forma parte de todo y está en todo al igual que Dios. La maldad puede estar frente a nosotros y no nos hará parte de ella, si contamos con la presencia de Dios, si estamos con Dios y Él con nosotros. Al huir de Dios, el mal nos consumirá y es imposible evitar esto si no estamos en sintonía con el Creador.

No soy partidario del concepto religioso de Dios, es interesante aunque limitada su comprensión debido al arraigo cultural y dogmático de las cosas. Dios es el Universo y el Universo también puede ser Dios, es algo más que la imagen viva de Dios en el panteísmo. No es una entidad como nosotros, aunque seamos a imagen y semejanza, sino algo como el todo y la nada a la vez, como una esencia. Es la flor que hemos elegido de entre miles de flores, algo la define como una flor y algo la define como única y distinguible de entre las demás. Por debajo de ella, hay algo que permite que sea lo que es y no sea lo que no es, lo que permite su existencia y evita su inexistencia. ¿Podría ser un barco el mismo luego de que paulatinamente se les cambiaran todas sus piezas y partes originales por otras totalmente nuevas debido al mantenimiento preventivo y correctivo que ha de llevar?

\end{document}
