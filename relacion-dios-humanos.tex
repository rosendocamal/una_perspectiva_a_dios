\documentclass[letterpaper,13pt]{article}

% Paquetes para formato y acentos
\usepackage[utf8]{inputenc}  % Codificación de entrada (acentos)
\usepackage[spanish]{babel}   % Idioma español
\usepackage{amsmath}          % Paquete para usar símbolos matemáticos
\usepackage{amsfonts}         % Paquete para fuentes matemáticas
\usepackage{amssymb}          % Paquete para símbolos adicionales
\usepackage{geometry}         % Ajustes de márgenes

% Ajuste de márgenes
\geometry{top=1in,bottom=1in,left=1in,right=1in}

\title{Análisis de la Relación entre Dios y el Ser Humano} % Título
\author{J. Rosendo Camal} % Autor
\date{Playa del Carmen, Quintana Roo, México a 29 de diciembre de 2024} % Fecha

\begin{document}

\maketitle  % Inserta el título, autor y fecha

\section{Introducción}

Mi análisis sobre la relación entre Dios y el ser humano proviene de una reflexión personal, sin formación formal en teología, filosofía o matemáticas, solo un principiante o aficionado. Aunque solo cuento con estudios de nivel medio superior, he buscado maneras de abordar temas tan profundos y abstractos recurriendo a métodos simples y analogías comprensibles, como la que incluyo sobre los números irracionales.

Las matemáticas utilizadas aquí no pretenden reducir lo divino a algo concreto, racional o humano, sino que son un medio para ilustrar mi comprensión. Tal como los físicos y matemáticos simplifican conceptos complejos al explicárselos a niños y al público en general. Aunque los divulgadores lo hagan mejor, también alguien simplificará y las hará más accesible de manera magistral. Trato de encontrar maneras de acercar estas ideas incomprensibles o abstractas a través de métodos accesibles, aunque no reflejen su esencia total.

Este análisis no es definitivo y está abierto al diálogo, reconociendo que presenta ciertas debilidades y amenazas, pero también fortalezas y oportunidades que podrían enriquecer futuras reflexiones.

\section{Dios y el Ser Humano: La Semejanza y la Distinción}

El relato bíblico menciona que Dios creó al ser humano a imagen y semejanza de Él. Esto implica, que tenemos aspectos similares, sin embargo, somos distintos de Él, por que Dios y solo Dios puede ser igual y ser Dios. Ser similar no es preciso a ser igual y ser igual dista de ser exactamente igual, al propio ser. La naturaleza de Dios es descrita como infinita y divina, y los seres humanos como finitos y limitados. Esta diferencia es la esencia que mantiene la relación entre el Creador y su Creación. Lo anterior implica para los seres humanos, que aunque tengamos parecido con Dios no somos iguales y a la vez mantenemos una semejanza a Él.

San Agustín refuerza esta relación cuando dice: ``Dios está más cerca de nosotros que nosotros de nosotros mismos''. Esto resalta la cercanía y semejanza que compartimos con Él, pero también la diferencia esencial que nos separa.

En palabras más precisas nosotros somos aproximadamente iguales a Dios y viceversa, no somos iguales y tampoco lograríamos ser iguales, aunque se mantiene abierto un camino hipotético de lograr una equivalencia a Dios. Algo más cercano a ser igual, lo más cercano posible. Aquello es diferente de lo que somos ahora, y nos permitiría ser equivalentes. Esa equivalencia marcaría el límite entre el Creador y su Creación, por que siendo equivalentes no seríamos exactamente iguales.

Un ejemplo cotidiano que ayuda a visualizar esto es el espejo. Refleja nuestra imagen, mostrando semejanzas, pero ese reflejo nunca será nosotros. La distancia entre el reflejo y la persona ilustra la relación entre los seres humanos y Dios: semejanza, pero no igualdad.

Un ejemplo de Dios, de lo distinto y semejante lo encontramos en el libro del Génesis. Dios ha demostrado su semejanza a nosotros y prueba de ello es el arrepentimiento de habernos creado cuando la maldad se esparció entre nosotros, en su creación. Los seres humanos también se arrepienten de sus actos, sean malos o buenos, correctos o incorrectos, se arrepiente por apartarse de Dios y un sinfín de motivos; y tratan de corregir su acto. Refleja lo semejante que somos a Dios y viceversa.

Enfatizo en que dicho acto no es una debilidad de Dios, sino su manifestación en su creación. Esto refleja lo distinto que somos a Dios y Él de nosotros. El arrepiento humano no es igual al arrepentimiento divino. El arrepentimiento divino implica que Dios se manifestó de manera dinámica en su creación, una demostración de su relación con nosotros y así mantener la armonía del plan divino, con la finalidad de evitar su ausencia.

\section{Lo Igual y lo Aproximadamente Igual: Una Analogía Matemática}

El conjunto de los números irracionales (I) poseen unas características particulares que lo diferencia de los demás conjuntos de números. Ejemplos de estos números son $\pi$, $e$, $\varphi$ y $\lambda$. Estos números no se pueden representar a través de fracciones y no hemos logrado hallar su representación numérica exacta, ya que su parte decimal es infinita y además no poseen un patrón para deducir su infinitud por completo o al menos nuestra comprensión actual no nos permite hallar dicho patrón.

La relación de Dios y los seres humanos, de forma más general entre el Creador y su Creación, se puede relacionar con una analogía matemática. Galileo Galilei afirmó que ``las matemáticas son el lenguaje con el que Dios escribió el universo.'' Las matemáticas tal vez sea el lenguaje del Creador o de su Creación, o quizás sea una interpretación humana del lenguaje divino o de la Creación; esto no lo puedo asumir pero es interesante recordarla de vez en cuando. Pero, si asumimos que esto es correcto, entonces las matemáticas nos brindan herramientas para entender al menos una parte de esta relación.

La analogía no es únicamente un ejercicio intelectual y simplista de la relación entre Dios y los seres humanos, es una manera de las distintas que hay y ha de haber de dar a conocer la relación entre el Creador y su Creación; y que refleja de una manera más precisa la naturaleza de la semejanza y distinción de esa relación.

Empecemos con la analogía. Representemos y definamos a Dios como $\pi$ y a los seres humanos como las siguientes aproximaciones más utilizadas y conocidas de $\pi$: 3.14 y 3.1416. Ambos valores que nos representan en este marco, los trataré de forma individual y paralela al mismo tiempo.

\[
\text{Dios} := \pi \quad
\]
\[
\text{Humanos} := 3.14; \quad \text{Humanos} := 3.1416
\]

Las aproximaciones de $\pi$ como 3.14 y 3.1416 son útiles porque son finitas y limitadas, y eso se asemeja a los atributos finitos y limitados propios de los humanos y de las cosas creadas. Mientras que $\pi$, representa lo divino y lo infinito, atributos de Dios. Cabe mencionar que $\pi \approx 3.14$ y $\pi \approx 3.1416$. Y si conservamos además del valor numérico, el significado, tenemos $3.14 \approx 3.1416$, de lo contrario sería $3.14 \neq 3.1416$. El significado no es más que el valor adicional, la estructura o la esencia que le añadimos a las cosas, en este caso, el valor como aproximaciones a $\pi$. Lo que permite nutrir a 3.14 y 3.1416 algo más que solo valores numéricos. Lo anterior, demuestra que tanto 3.14 y 3.1416 son aproximadamente iguales ($\approx$) a $\pi$ y entre sí. Tal cual como el Creador con la Creación y las cosas creadas entre sí.

No son exactamente iguales ($=$), sino aproximadamente iguales ($\approx$). Porque si bien 3.14 y 3.1416 son casi iguales a $\pi$, son distintos a dicho número y entre sí. El significado que le hemos otorgado y la relación entre sí permite que sean casi iguales, pero no exactamente iguales. Por lo que tenemos $\pi \approx 3.14 \approx 3.1416$; $\pi \neq 3.14$, $\pi \neq 3.1416$ y también $3.14 \neq 3.1416$. El valor numérico y el significado permite recalcar que son aproximadamente iguales y distintos a la vez.

\[
\pi \neq 3.14 \neq 3.1416
\]
\[
(=) \neq (\approx)
\]

Si lográsemos hallar el valor numérico exacto de $\pi$ y los comparamos entre sí obtendríamos una equivalencia ($\equiv$). Ese valor que aún no ha sido hallado y está en espera de ello, lo podríamos representar como 3.1415... y esto, aunque no es lo más preciso posible, es de utilidad en esta analogía.

\[
\pi \equiv 3.1415...
\]
\[
\pi \neq 3.1415...
\]

Las dos expresiones anteriores son posibles porque aunque $3.1415\ldots$ sea el valor numérico exacto de $\pi$, es distinto de él. Son equivalentes ($\equiv$), esto es una escala más arriba de lo que es ser aproximadamente igual ($\approx$) y a pesar de ello ambos aún están demasiado por debajo de ser exactamente iguales ($=$). $\pi$ es algo más que un simple valor exacto, es una constante y una relación geométrica, mientras que $3.1415\ldots$ es sólo un valor numérico exacto de $\pi$ y no es más que la representación numérica de dicha constante. No llega a la esencia de ser $\pi$, ser exactamente igual a $\pi$ y ser como $\pi$. No lo alcanzaría, nuestras representaciones numéricas pasarían de ser aproximadamente iguales o casi iguales a ser equivalentes o iguales más no exactamente iguales. Lo aproximadamente igual es distinto de lo equivalente y de lo exactamente igual, y los tres son distintos entre sí.

\[
(\approx) \neq (\equiv)
\]
\[
(=) \neq (\equiv)
\]
\[
(=) \neq (\equiv) \neq (\approx)
\]

De igual modo, afirmo que los seres humanos actualmente somos a imagen y semejanza de Dios -aproximadamente iguales- y quizás (en un hipotético caso aún no demostrado) podremos acercarnos aún más a la naturaleza divina, al terreno de Dios y, sin embargo, a pesar de ello jamás seremos iguales a Él. Porque solo Dios es exactamente igual a Dios y solo Dios es Dios. La aproximación ($\approx$) y la equivalencia ($\equiv$) las podemos reducir a la semejanza ($\sim$), que por definición es distinto de la igualdad divina ($=$) y que además, incluye al menos dos formas de acercarnos de una manera más directa con el Creador. Por lo que a Él le corresponde la igualdad divina y a su creación la semejanza ($\sim$). Cabe aclarar, Dios también puede usar la semejanza si no, no podría ser semejante a nosotros y nosotros de Él, lo usa a su favor para interactuar con la creación que está hecho a base de la semejanza.

\[
(\sim) = \{ (\approx), (\equiv) \}
\]
\[
(=) = \{ (=) \}
\]

Retomando la analogía de $\pi$, solo $\pi$ puede ser $\pi$ de forma plena y no existe otro elemento que sea $\pi$. Y esto lo podemos representar en una expresión lógica:

\[
\exists!x \left( x = \pi \right) \land \left( \forall y \left( y = \pi \rightarrow y = x \right) \right)
\]

La expresión anterior -simplificando a $\exists!x \left( x = \pi \right) \land \left( \pi = \pi \right)$-, indica que $\pi$ es único y no puede ser igual a ninguno de sus semejantes. Del mismo modo, Dios es único y solo Él es exactamente igual a sí mismo y Él es él mismo, y no puede ser exactamente igual a su creación, aunque sea su imagen y reflejo.

\[
C(x): \, x \text{ es una cosa creada (pertenece a la creación).}
\]
\[
D(y): \, y \text{ es el Creador.}
\]
\[
x \sim y: \, x \text{ es semejante a } y.
\]
\[
x = y: \, x \text{ e } y \text{ tienen igualdad divina (igual pura).}
\]
\[
x \neq y: \, x \text{ e } y \text{ no son iguales (no tienen igualdad divina).}
\]

\[
\forall x \forall y \left( C(x) \land D(y) \rightarrow (x \sim y \land x \neq y) \right) \land (y = y)
\]

\section{Conclusión: La Semejanza como Puente entre Dios y los Seres Humanos y de las Cosas Creadas}

En resumen, la relación entre Dios y los seres humanos y las cosas creadas está dada exclusivamente en la semejanza ($\sim$), lo que establece una conexión entre ambos, pero sin olvidar su distinción. La igualdad divina ($=$) pertenece únicamente al Creador, quien es igual a sí mismo y único en esencia. Esto preserva la separación entre el Creador y su Creación: el Creador, Dios, permanece infinito, divino y único, mientras que las cosas creadas participan de su semejanza pero nunca de la igualdad. La semejanza se convierte en un puente entre ambos. Karl Barth, al señalar distinción entre el Creador y la Creación, permite reforzar que, aunque los humanos podamos reflejar aspectos divinos, esa diferencia esencial seguirá siendo inquebrantable.

\end{document}
